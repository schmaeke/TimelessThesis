\chapter{Example Chapter with random title}
\blindtext[3]

\section{First subsection}
\blindtext[4]
\begin{equation}
	\begin{aligned}
		\int \limits_\Omega \bs{\varepsilon}(\delta\bs{u}) : \bs{C} : \bs{\varepsilon}(\bs{u})\intd{\Omega}
		&= \int \limits_\Omega \delta\bs{u} \bs{p}\intd{\Omega}
		 + \int \limits_{\Gamma_N} \delta\bs{u} \bs{q}\intd{\partial\Omega}
		 + \int \limits_{\Gamma_D} \delta\bs{u} \bs{q}\intd{\partial\Omega} \\
		\land \quad \bs{u} &= \overline{\bs{u}} \quad \forall \bs{x} \in \Gamma_D \\
	\end{aligned}
\end{equation}
\blindtext[2]
\begin{figure}
	\begin{center}
		\input{content/example/images/example_tikz_image.tex}
	\end{center}
	\caption{This is an example image. The visualization has been done using the \texttt{TikZ} package.}
	\label{fig:example_tikz_image}
\end{figure}
\subsection{First subsection}
\Cref{fig:example_tikz_image} \blindtext[3]
\subsubsection{First subsubsection}
\blindtext[3]
\subsubsection{Second subsubsection}
\blindtext[3]
\subsection{Second subsection}
\blindtext[4]
\begin{figure}
	\begin{center}
		\begin{tikzpicture}[spy using outlines={black, line width = 1, densely dashed, thick, rectangle, size=2cm, magnification=1.6, connect spies}]

	\begin{axis} [
		font={\footnotesize},
		axis lines = box,
		xlabel = $x$,
		ylabel = $y$,
		width = 0.75\textwidth,
		height = 8cm,
		max space between ticks = 40,
		grid = major,
		grid style = {densely dashed, line width = 0.1pt},
		minor x tick num = 9,
		minor y tick num = 9,
		cycle list name = veluraColorList,
		legend pos = north east,
		legend cell align={left},
		domain = -10:10,
		smooth,
	]

		\addplot { -1.0 * atan( 1.0 * x ) };
		\addplot { -1.1 * atan( 1.1 * x ) };
		\addplot { -1.2 * atan( 1.2 * x ) };
		\addplot { -1.3 * atan( 1.3 * x ) };
		\addplot { -1.4 * atan( 1.4 * x ) };
		\addplot { -1.5 * atan( 1.5 * x ) };
		\addplot { -1.6 * atan( 1.6 * x ) };
		\addplot { -1.7 * atan( 1.7 * x ) };
		\addplot { -1.8 * atan( 1.8 * x ) };

		\legend{
			$\alpha = 1.0$,
			$\alpha = 1.1$,
			$\alpha = 1.3$,
			$\alpha = 1.4$,
			$\alpha = 1.5$,
			$\alpha = 1.6$,
			$\alpha = 1.7$,
			$\alpha = 1.8$,
		};

	\end{axis}

	\spy[height = 2.5cm, width = 2.5cm] on (2.5, 5.2) in node[fill = white] at (1.5, 1.5);

\end{tikzpicture}

	\end{center}
	\caption{This is an example plot. The visualization has been done using the \texttt{pgfplots} package.}
	\label{fig:example_tikz_image}
\end{figure}
\cite{bathe2007finite} \blindtext[5]

\begin{lstlisting}[caption = {Example of 'lstlisting' with Julia code}, label = {code:example}, float]
function some_function( dimension::Int64 )
	return "Hello World"
end # function
\end{lstlisting}

\todo{Example of a 'TODO'. These can be placed during writing to add comments on open tasks.}
